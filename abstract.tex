Large supercomputers present several challenges in effectively programming parallel applications: exposing concurrency, optimizing data movement, controlling load imbalance, addressing heterogeneity, handling variations in application's behavior, and tolerating system failures.
Addressing these challenges requires an emphasis on important
concepts during application development: overdecomposition, asynchrony, migratability,
and adaptivity. To realize all those features, the runtime systems will need to become
introspective and provide automated support for several tasks that currently burden the programmer.
This tutorial is aimed at exposing the attendees to the aforementioned concepts. We
will present details on how a concrete implementation of these concepts, in synergy
with an introspective runtime system, can lead to development of applications that scale
irrespective of the rough landscape. We will focus on Charm++ as the
programming paradigm that encapsulates these ideas.
Charm++ provides an asynchronous, message-driven programming model via parallel
objects and an adaptive runtime system that guides execution. It automatically overlaps communication and computation,
balances loads, tolerates failures, checkpoints for split-execution, and
promotes modularity while allowing programming in C++. Several widely used
Charm++ applications thrive in computational science domains including biomolecular
modeling, cosmology, quantum chemistry, epidemiology, and  stochastic optimization. The approach followed in this tutorial is to provide a guide for migrating applications from the reigning parallel programming paradigm (MPI) to Charm++.
